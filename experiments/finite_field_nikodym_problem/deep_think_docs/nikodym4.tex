\documentclass{article}
\usepackage{amsmath}
\usepackage{amssymb}
\usepackage{amsthm}

% --- Environment Definitions ---
\newtheorem{theorem}{Theorem}[section]
\newtheorem{lemma}[theorem]{Lemma}
\theoremstyle{definition}
\newtheorem{definition}[theorem]{Definition}

% --- Math Operators ---
\DeclareMathOperator{\deg_}{deg}

\begin{document}

This document provides a detailed and rigorous exposition of the proof adapting the anisotropic polynomial method, introduced by Bukh and Chao, to derive lower bounds for the size of Nikodym sets in finite fields.

\section{Preliminaries}

Let $\mathbb{F}_q$ be the finite field with $q$ elements. We work in the $n$-dimensional vector space $\mathbb{F}_q^n$. We assume $n \ge 3$ and $q \ge 2$.

\begin{definition}[Nikodym Set]
A set $N \subseteq \mathbb{F}_q^n$ is a Nikodym set if for every point $p \in \mathbb{F}_q^n$, there exists a line $L_p$ passing through $p$ such that $L_p \setminus \{p\} \subseteq N$.
\end{definition}

\begin{definition}[Multiplicity]
A polynomial $P \in \mathbb{F}_q[x_1, \dots, x_n]$ vanishes at a point $p \in \mathbb{F}_q^n$ with multiplicity at least $m$ if all partial derivatives of $P$ of order strictly less than $m$ vanish at $p$.
\end{definition}

We will utilize multiplicity $m=2$. $P$ vanishes at $p$ with multiplicity at least 2 if $P(p) = 0$ and all first-order partial derivatives $\partial_i P(p) = 0$ for $i=1, \dots, n$. This imposes $n+1$ linear constraints on the coefficients of $P$.

\begin{theorem}[Polynomials Vanishing on $\mathbb{F}_q^n$]
If a polynomial $P \in \mathbb{F}_q[x_1, \dots, x_n]$ vanishes on all of $\mathbb{F}_q^n$ and the individual degree $\deg_{x_i}(P) < q$ for all $i=1, \dots, n$, then $P$ is the zero polynomial $(P \equiv 0)$.
\end{theorem}

\section{The Anisotropic Polynomial Space}

We define a specific vector space of polynomials. We set the maximum total degree $D$ such that it is just below the number of zeros guaranteed on a Nikodym line segment with multiplicity 2, i.e., $D < 2(q-1)$. We set $D = 2q - 3$.

\begin{definition}[Anisotropic Space V]
We define the vector space $V \subseteq \mathbb{F}_q[x_1, \dots, x_n]$ as:
$$ V = \{P : \text{deg}(P) \le 2q - 3 \text{ and } \deg_{x_i}(P) < q \text{ for } i=1, \dots, n-1 \}. $$
This space is anisotropic as the constraint on $x_n$ is relaxed compared to the others.
\end{definition}

\section{The Key Lemma}

The core of the argument lies in the following lemma.

\begin{lemma}
Let $N$ be a Nikodym set in $\mathbb{F}_q^n$. If a polynomial $P \in V$ vanishes with multiplicity at least 2 at every point of $N$, then $P \equiv 0$.
\end{lemma}

\begin{proof}
\textbf{Step 1: P vanishes on the entire space $\mathbb{F}_q^n$.}

Let $p \in \mathbb{F}_q^n$ be arbitrary. By Definition 1.1, there exists a line $L_p$ through $p$ such that $S_p = L_p \setminus \{p\} \subseteq N$. Note that $|S_p| = q-1$.

Consider the restriction of $P$ to $L_p$, denoted $P|_{L_p}$. This is a univariate polynomial. Its degree is bounded by the total degree of $P$: $\deg(P|_{L_p}) \le \deg P \le 2q-3$.

By hypothesis, $P$ vanishes with multiplicity at least 2 at every point in $N$, and thus at every point in $S_p$. The total number of zeros of $P|_{L_p}$, counted with multiplicity, is at least $2 \cdot |S_p| = 2(q-1) = 2q-2$.

Since the number of zeros $(2q-2)$ exceeds the degree $(\le 2q-3)$, the polynomial $P|_{L_p}$ must be identically zero. In particular, $P(p) = 0$. As $p$ was arbitrary, $P$ vanishes everywhere on $\mathbb{F}_q^n$.

\textbf{Step 2: Polynomial Reduction.}

We utilize the anisotropic structure of $V$. We perform polynomial division of $P$ by $(x_n^q - x_n)$ with respect to the variable $x_n$.
$$ P(x) = (x_n^q - x_n)Q(x) + R(x), $$
where $Q(x)$ is the quotient and $R(x)$ is the remainder, satisfying $\deg_{x_n} R < q$.

We evaluate this equation at any point $a \in \mathbb{F}_q^n$. From Step 1, $P(a) = 0$. By Fermat's Little Theorem, $(a_n^q - a_n) = 0$. Thus, $0 = 0 \cdot Q(a) + R(a)$.
Thus, $R(a)=0$. The remainder $R$ vanishes on all of $\mathbb{F}_q^n$.

Now we examine the individual degrees of $R$. Since $P \in V$, $\deg_{x_i} P < q$ for $i=1, \dots, n-1$. Polynomial division with respect to $x_n$ does not increase the degrees with respect to the other variables, so $\deg_{x_i} R < q$ for $i=1, \dots, n-1$.

We now have a polynomial $R$ such that $\deg_{x_i} R < q$ for all $i=1, \dots, n$, and $R$ vanishes on $\mathbb{F}_q^n$. By Theorem 1.3, $R \equiv 0$.
We conclude that $P(x) = (x_n^q - x_n)Q(x)$.

\textbf{Step 3: Analyzing the Quotient Q.}

We determine the degree of $Q$.
$\deg P = \deg(x_n^q - x_n) + \deg Q = q + \deg Q$.
Since $\deg P \le 2q - 3$, we have $q + \deg Q \le 2q - 3$, which implies $\deg Q \le q - 3$.

\textbf{Step 4: Using the Derivative Constraints.}

We use the hypothesis that the derivatives of $P$ vanish on $N$. We examine the partial derivative with respect to $x_n$.
First, note that the formal derivative $\partial_n(x_n^q - x_n) = qx_n^{q-1} - 1$. In $\mathbb{F}_q$, this is equal to $-1$.

Differentiating $P(x)$ using the product rule:
$$ \partial_n P(x) = \partial_n(x_n^q - x_n) \cdot Q(x) + (x_n^q - x_n) \cdot \partial_n Q(x) $$
$$ \partial_n P(x) = (-1) \cdot Q(x) + (x_n^q - x_n) \cdot \partial_n Q(x) $$

Let $p \in N$. By hypothesis, $P$ vanishes to order 2 at $p$, so $\partial_n P(p) = 0$. Also, $(x_n^q - x_n)$ evaluated at $p$ is 0. Evaluating the equation at $p$:
$$ 0 = (-1) \cdot Q(p) + 0. $$
$$ 0 = -Q(p). $$
Thus, $Q(p) = 0$. The polynomial $Q$ vanishes on the entire set $N$.

\textbf{Step 5: Conclusion.}

We have established that $Q$ vanishes on $N$ and $\deg Q \le q - 3$.
Let $p$ be an arbitrary point in $\mathbb{F}_q^n$, and let $L_p$ be its Nikodym line. The restriction $Q|_{L_p}$ has degree $\le q-3$. It vanishes on the $q-1$ points in $L_p \setminus \{p\}$.
Since the number of zeros $(q-1)$ exceeds the degree $(\le q-3)$, $Q|_{L_p} \equiv 0$. Thus $Q(p) = 0$.

$Q$ vanishes on all of $\mathbb{F}_q^n$. Since $\deg Q < q$, all individual degrees are less than $q$. By Theorem 1.3, $Q$ is the zero polynomial $(Q \equiv 0)$.
Consequently, $P(x) = (x_n^q - x_n)Q(x) \equiv 0$.
\end{proof}

\section{Dimension Calculation and Lower Bound}

We now calculate the dimension of $V$. We use the Principle of Inclusion-Exclusion to count the number of monomials satisfying the degree constraints: $\sum_{i=1}^n e_i \le D$ and $e_i < q$ for $i=1, \dots, n-1$.
$$ \dim V = \sum_{S \subseteq \{1, \dots, n-1\}} (-1)^{|S|} \binom{D-|S|q+n}{n}. $$
Substituting $D = 2q-3$:
$$ \dim V = \binom{2q-3+n}{n} - \binom{n-1}{1}\binom{2q-3-q+n}{n} + \binom{n-1}{2}\binom{2q-3-2q+n}{n} - \dots $$
$$ \dim V = \binom{2q+n-3}{n} - (n-1)\binom{q+n-3}{n} + \binom{n-1}{2}\binom{n-3}{n} - \dots $$
Since we assume $n \ge 3$, we have $n-3 \ge 0$. As the upper index $n-3$ is strictly less than the lower index $n$, the binomial coefficient $\binom{n-3}{n} = 0$. All subsequent terms involve even smaller upper indices and thus are also 0.

Thus, for $n \ge 3$:
$$ \dim V = \binom{2q+n-3}{n} - (n-1)\binom{q+n-3}{n}. $$

\begin{theorem}
Let $N$ be a Nikodym set in $\mathbb{F}_q^n$ ($n \ge 3, q \ge 2$). Then,
$$ |N| \ge \frac{1}{n+1}\left(\binom{2q+n-3}{n} - (n-1)\binom{q+n-3}{n}\right). $$
\end{theorem}

\begin{proof}
We consider the linear constraints imposed on the space $V$ by requiring that a polynomial $P \in V$ vanishes to order 2 at every point $p \in N$. This yields a total of $(n+1)|N|$ linear constraints.

If the number of constraints is less than the dimension of the space, $(n+1)|N| < \dim V$, then there must exist a non-zero polynomial $P \in V$ satisfying these constraints. However, Lemma 3.1 shows that any such non-zero polynomial must be the zero polynomial. This is a contradiction.

Therefore, we must have:
$$ (n+1)|N| \ge \dim V. $$
Rearranging and substituting the expression for $\dim V$:
$$ |N| \ge \frac{\dim V}{n+1} = \frac{1}{n+1}\left(\binom{2q+n-3}{n} - (n-1)\binom{q+n-3}{n}\right). $$
\end{proof}

\section{Asymptotic Analysis}

We analyze the asymptotic behavior as $q \to \infty$ for fixed $n$. We use the approximation $\binom{Aq+B}{n} = \frac{(Aq)^n}{n!} + O(q^{n-1})$.
$$ \dim V = \frac{(2q)^n}{n!} + O(q^{n-1}) - (n-1)\left(\frac{q^n}{n!} + O(q^{n-1})\right) $$
$$ \dim V = \frac{q^n}{n!}(2^n - (n-1)) + O(q^{n-1}) = \frac{q^n}{n!}(2^n - n + 1) + O(q^{n-1}). $$
The lower bound is $|N| \ge C_n q^n + O(q^{n-1})$, where the constant $C_n$ is:
$$ C_n = \frac{2^n-n+1}{(n+1) \cdot n!} = \frac{2^n-n+1}{(n+1)!}. $$

\end{document}
