%%%%%%%%%%%%%%%%%%%%%%%%%%%%%%%%%%%%%%%%%%%%%%%%%%%%%%%%%%%%%%%%%%%%
% A LaTeX document for the proof of the size of a Nikodym set
% constructed via the First Moment Method.
%
% To compile: pdflatex nikodym_proof.tex
%%%%%%%%%%%%%%%%%%%%%%%%%%%%%%%%%%%%%%%%%%%%%%%%%%%%%%%%%%%%%%%%%%%%

\documentclass{article}

% --- PACKAGES ---
\usepackage[a4paper, margin=1in]{geometry} % Sensible page margins
\usepackage{amsmath}                       % For advanced math environments
\usepackage{amssymb}                       % For more math symbols (like \mathbb)
\usepackage{amsfonts}                      % For math fonts
\usepackage{amsthm}                        % For proof environments

% --- THEOREM-LIKE ENVIRONMENTS ---
\newtheorem{theorem}{Theorem}
\newtheorem{lemma}[theorem]{Lemma}
\newtheorem{proposition}[theorem]{Proposition}
\theoremstyle{definition}
\newtheorem{definition}{Definition}

% --- DOCUMENT INFORMATION ---
\title{Construction of a Nikodym Set via the First Moment Method}
\author{Proof generated by Gemini}
\date{\today}

% --- BEGIN DOCUMENT ---
\begin{document}

\maketitle

\begin{abstract}
We demonstrate the existence of a large Nikodym set in $F_p^3$ for a large prime $p$ using a purely probabilistic argument. By applying the first moment method, we show that a random set of points is a Nikodym set with high probability, provided its density is sufficiently high. This approach allows us to construct a Nikodym set of size approximately $p^3 - 2p^2\log(p)$.
\end{abstract}

\section{Introduction}

Let $F_p$ be the finite field with $p$ elements, where $p$ is a large prime. We are working in the vector space $F_p^3$.

\begin{definition}[Nikodym Set]
A set $N \subseteq F_p^3$ is called a \textbf{Nikodym set} if for every point $x \in F_p^3$, there exists a line $L$ passing through $x$ such that the set of all other points on that line, $L \setminus \{x\}$, is fully contained in $N$.
\end{definition}

We will construct such a set using the first moment method. The "best size" will be interpreted as the largest possible Nikodym set with a non-trivial complement that this method can produce.

\section{The Probabilistic Construction}

The proof relies on the following probabilistic setup:
\begin{enumerate}
    \item Create a random subset $S \subseteq F_p^3$ by including each of the $p^3$ points independently with probability $\pi$.
    \item A point $x \in F_p^3$ is defined as \textbf{``bad''} if it fails to satisfy the Nikodym condition with respect to $S$. That is, for every line $L$ passing through $x$, $L \setminus \{x\} \not\subset S$. Let $B$ be the random variable representing the set of all bad points.
    \item The first moment method states that if the expected number of bad points is less than 1, i.e., $E[|B|] < 1$, then there is a non-zero probability that $|B| = 0$. This implies the existence of at least one specific set $S$ with no bad points, which is therefore a Nikodym set.
\end{enumerate}

Our goal is to find the smallest probability $\pi$ (and thus the largest complement $p^3(1-\pi)$) that satisfies the condition $E[|B|] < 1$.

\section{The Proof}

\begin{theorem}
For a sufficiently large prime $p$, there exists a Nikodym set $N \subset F_p^3$ of size at least $p^3 - 2p^2\log(p) - O(p^2)$.
\end{theorem}

\begin{proof}
The proof proceeds in three main steps.

\subsection{Calculating the Expectation}
First, we compute the expected number of bad points, $E[|B|]$.
The number of lines passing through any point $x \in F_p^3$ is $q = \frac{p^3-1}{p-1} = p^2+p+1$. For large $p$, we can approximate this as $q \approx p^2$.

For any single line $L$ through $x$, the set $L \setminus \{x\}$ contains $p-1$ points. The probability that all of these points are in $S$ is $\pi^{p-1}$. The probability that this condition fails for line $L$ is therefore $1-\pi^{p-1}$.

A point $x$ is bad if this condition fails for all $q$ lines passing through it. Since the sets $L_i \setminus \{x\}$ for distinct lines $L_i$ are disjoint, these events are independent. The probability that a point $x$ is bad is:
$$ P(x \text{ is bad}) = (1 - \pi^{p-1})^q $$
By linearity of expectation, the expected total number of bad points is:
$$ E[|B|] = \sum_{x \in F_p^3} P(x \text{ is bad}) = p^3 (1 - \pi^{p-1})^q $$

\subsection{Applying the Constraint}
The condition for the first moment method to guarantee existence is $E[|B|] < 1$, which gives the inequality:
$$ p^3 (1 - \pi^{p-1})^q < 1 $$
We wish to find the smallest $\pi$ that satisfies this. Let's parameterize $\pi$ as follows, where $c$ is a constant we want to maximize:
$$ \pi = 1 - \frac{c \log p}{p} $$
Substituting this into the inequality, we first approximate the term $\pi^{p-1}$:
$$ \pi^{p-1} = \left(1 - \frac{c \log p}{p}\right)^{p-1} \approx \exp\left(-\frac{c \log p}{p} \cdot (p-1)\right) \approx e^{-c \log p} = p^{-c} $$
Now we can analyze the full expression for $E[|B|]$, using the inequality $1-z < e^{-z}$ for $z>0$:
$$ E[|B|] = p^3 (1 - \pi^{p-1})^q \approx p^3 (1 - p^{-c})^{p^2} < p^3 \left(e^{-p^{-c}}\right)^{p^2} = p^3 e^{-p^2 \cdot p^{-c}} = p^3 e^{-p^{2-c}} $$
For the method to succeed, we need this upper bound to be less than 1.
$$ p^3 e^{-p^{2-c}} < 1 $$
Taking the natural logarithm of both sides gives:
$$ \ln(p^3) - p^{2-c} < 0 \implies 3 \log p < p^{2-c} $$

\subsection{Determining the Constant c}
We analyze the inequality $3 \log p < p^{2-c}$ to find the valid range for $c$.
\begin{itemize}
    \item If \textbf{c < 2}, then $2-c > 0$. The term $p^{2-c}$ grows as a positive power of $p$, while $3 \log p$ grows only logarithmically. For any $c<2$, the inequality will hold for all sufficiently large $p$.
    \item If \textbf{c = 2}, the inequality becomes $3 \log p < p^0 = 1$, which is false for $p \ge 2$.
    \item If \textbf{c > 2}, then $2-c < 0$. The term $p^{2-c} \to 0$ as $p \to \infty$. The inequality $3 \log p < 0$ is false.
\end{itemize}
Thus, the first moment method succeeds for any choice of $c < 2$.

\section{Conclusion}
The proof shows that a random set $S$ with point inclusion probability $\pi = 1 - \frac{c \log p}{p}$ is a Nikodym set with positive probability for any constant $c < 2$.

The size of this set is given by:
$$ |S| = p^3 \pi \approx p^3 \left(1 - \frac{c \log p}{p}\right) = p^3 - c p^2 \log p $$
To get the largest possible complement, we want to maximize $c$. Since we can choose $c$ to be any value arbitrarily close to 2, the method proves the existence of a Nikodym set of size at least $p^3 - 2 p^2 \log(p) - O(p^2\log(\log(p)))$. Ignoring the lower order terms, the best size we can achieve is approximately:
$$ p^3 - 2 p^2 \log(p) $$
This completes the proof.
\end{proof}

\end{document}
% --- END DOCUMENT ---
