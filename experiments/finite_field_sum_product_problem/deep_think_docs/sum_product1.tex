\documentclass{article}
\usepackage{amsmath, amssymb, amsfonts}
\usepackage{geometry}
\geometry{a4paper, margin=1in}

\title{A Sum-Product Construction in $\mathbb{F}_p$ via Gaussian Integers}
\author{A.I. Assistant}
\date{\today}

\begin{document}

\maketitle

\section*{The Construction}

Let $p$ be a prime such that $p \equiv 1 \pmod 4$. We describe a construction for a set $A \subset \mathbb{F}_p$ of size $k = \lfloor\sqrt{p}\rfloor$ aiming to minimize both $|A+A|$ and $|A \cdot A|$.

The method is based on the ring of Gaussian integers, $\mathbb{Z}[i]$. The condition $p \equiv 1 \pmod 4$ ensures the existence of an element $i_p \in \mathbb{F}_p$ such that $i_p^2 \equiv -1 \pmod p$. This induces a natural ring homomorphism $\phi: \mathbb{Z}[i] \to \mathbb{F}_p$ given by $\phi(a+bi) = a + b i_p \pmod p$.

The set $A$ is the image under $\phi$ of a subset of $\mathbb{Z}[i]$ whose elements $z = a+bi$ are chosen to be ``small'' in both a multiplicative and an additive sense.

\subsection*{Selection Criterion}

A large pool of candidate elements $z = a+bi$ is generated by searching over integer coordinates $(a,b)$ in a circular region, e.g., $a^2+b^2 \le R^2$ for some radius $R$ proportional to $\sqrt{k}$. These candidates are then sorted to find the most suitable elements.

The primary sorting key is the \textbf{largest prime factor (LPF)} of the norm $N(z) = a^2+b^2$.
\begin{itemize}
    \item \textbf{Multiplicative Structure}: By selecting elements $z$ for which $N(z)$ is a smooth number (has a small LPF), we impart a strong multiplicative structure. Since the norm is multiplicative, $N(z_1 z_2) = N(z_1)N(z_2)$, the norms of products of our chosen elements will also be composed of small primes. This structure is intended to constrain the size of the product set $A \cdot A = \{\phi(z_j z_l)\}$.
    \item \textbf{Additive Structure}: As a secondary tie-breaking criterion, we minimize the $L_\infty$-norm, $\|z\|_\infty = \max(|a|,|b|)$. Choosing elements with small coordinates is intended to control the size of the sum set $A+A$, as the sums of these coordinates will be confined to a smaller region of the integer lattice.
\end{itemize}
The full sorting order for a candidate $z=a+bi$ is based on the tuple $(\text{LPF}(N(z)), N(z), \|z\|_\infty)$.

\subsection*{Final Set}
The set $A$ is formed by taking the top $k$ candidates from the sorted list, applying the homomorphism $\phi$, and ensuring $k$ unique elements are collected in $\mathbb{F}_p$. In essence, the construction identifies elements in $\mathbb{Z}[i]$ that are compact in the integer lattice and possess arithmetically simple norms, then maps them into the finite field. For $p = 1,000,000,009$, the resulting sum and product sets $A+A$ and $A \cdot A$ have sizes less than $|A|^{1.47}$.

\end{document}
